\documentclass[12pt,a4paper]{article}
\usepackage[utf8]{inputenc}
\usepackage[russian]{babel}
\usepackage[OT1]{fontenc}
\usepackage{amsmath}
\usepackage{amsfonts}
\usepackage{amssymb}
\usepackage{graphicx}
\graphicspath{{Images/}}
\usepackage[left=2cm,right=2cm,top=2cm,bottom=2cm]{geometry}
\usepackage{calc}
\usepackage{wrapfig}
\usepackage{setspace}
\usepackage{indentfirst}
\usepackage{amssymb}
\usepackage{subfigure}
\usepackage{multirow}

\title{
Отчет о выполнении лабораторной работы 3.1.3

Измерение магнитного поля земли 
}

\author{Комкин Михаил, группа Б01-303}
\newpage
\begin{document}

\maketitle

\textbf{Цель работы:} исследовать свойства постоянных неодимовых магнитов; измерить с их помощью горизонтальную и вертикальную составляющие индукции магнитного 
поля Земли и магнитное наклонение.

\textbf{В работе используются:} неодимовые магниты; тонкая нить для изготов- ления крутильного маятника; медная проволока; электронные весы; секун- домер; измеритель 
магнитной индукции; штангенциркуль; брусок, линейка и штатив из немагнитных материалов; набор гирь и разновесов.


\section{Свойства точечного магнитного диполя}
Простейший магнитный диполь может быть образован витком с током или постоянным магнитом. По определению, магнитный момент $m$ тонкого витка площадью 
$S$ с током $I$ равен (в системе СИ)
\begin{equation}
    m = IS
\end{equation}
\label{moment}

где $S = S_n$ — вектор площади контура, образующий с направлением тока правовинтовую систему, п единичный вектор нормали к площадке (это же направление $m$ принимается 
за направление $S \to N$ от южного $Ѕ$ к северному N полюсу магнита). Если размеры контура с током или магнитной стрелки малы по сравнению с расстоянием до диполя, 
то соответствующий магнитный диполь m называют элементарным, или точечным.
Магнитное поле точечного диполя определяется по формуле, аналогичной формуле для поля элементарного электрического диполя:
\begin{equation}
    B_\text{дин} = \frac{\mu_0}{4 \pi} \left(\frac{3(m \cdot r)r}{r^5} - \frac{m}{r^3}\right)
\end{equation}
\label{dipol}
(здесь $\mu/4\pi = 10^{-7} \text{Гн/м}$).    
    Во внешнем магнитном поле с индукцией $B$ на точечный магнитный диполь $m$ действует механический момент сил 
\begin{equation}
    M = [m \times B]
\end{equation} 
\label{mehmoment}

При этом потенциальная энергия, которой обладает диполь с постоянным $m$, равна

\begin{equation}
    W = - (m \cdot B).
\end{equation}
\label{energy}


Когда диполь ориентирован вдоль внешнего поля ($m || B$), он находится в состоянии равновесия ($М= 0$). При этом устойчивым будет только состояние, в котором диполь 
сонаправлен с полем $m || В$, поскольку его потенциальная энергия достигает минимума ($W_{min} = -mB$). При противоположной ориентации энергия будет иметь максимум 
($W_{max} = mB$), и состояние равновесия будет неустойчивым.

    В неоднородном внешнем поле выражение для энергии постоянного диполя (4) сохраняется. При этом кроме момента сил на диполь действует еще и сила 
\begin{equation}
    F = - \nabla W = - (m \cdot \nabla)B
\end{equation}
\label{sila}
Таким образом, из (3)-(5) следует, что свободный магнитный диполь в неоднородном магнитном поле ориентируется вдоль силовых линий магнитного поля и втягивается в область более сильного поля, поскольку это ведёт к уменьшению энергии диполя.
Выражения (2) и (5) позволяют рассчитать силу взаимодействия маг- нитов с моментами m1 и m2 в рамках модели точечных диполей. В част- ном случае, когда моменты двух небольших магнитов направлены вдоль
соединяющей их прямой: $m_{1,2} || r$, где $r$ радиус-вектор между ними, магниты взаимодействуют с силой

\begin{equation}
    F_{12} = m_1 \frac{\partial B_2}{\partial r} = m_1 \frac{\partial(2m_2/r^3)}{\partial r} = \frac{6m_1 m_2}{r^4} (\text{ед. СГС})
\end{equation}
\label{six}

Здесь магниты притягиваются, если их магнитные моменты сонаправлены, и отталкиваются, если направлены противоположно.
    Если магнитные моменты направлены перпендикулярно соединяющей их прямой, то сила взаимодействия окажется в два раза меньшей и будет иметь противоположный знак.
\[
F_{12} = \frac{3m_1 m_2}{r^4} (\text{ед. СГС})
\]

\section{Экспериментальная установка}

    В работе используются неодимовые магниты шарообразной формы.
    Для проведения эксперимента важно, что а) вещество, из которого изготовлены магниты, является магнитожёстким материалом; б) шары намагничены однородно.
    «Магнитожёсткость» материала означает, что магнитные моменты шаров в процессе работы не изменяются под действием внешних магнитных полей, 
    Т. е. шар ведёт себя как постоянный («жёсткий») диполь. В том числе, магнитные моменты не изменяются при контакте магнитов друг с другом.
Магнитное поле однородно намагниченного шара радиусом R может быть вычислено точно. На расстояниях $r >= R$ от центра шара оно совпа- дает с полем точечного 
магнитного диполя (2), расположенного в центре, магнитный момент $m$ которого совпадает с полным моментом шара. Внутри шара магнитное поле однородно: с 
помощью формулы (2) и условия непрерывности нормальной компоненты индукции на поверхности шара нетрудно получить, что при $r < R$
\begin{equation}
    B_0 = \frac{\mu_{0} m}{2 \pi R^3}
\end{equation}
\label{pole}
(в ед. СГС $B_0 = 2m/R^3$)


    В качестве ещё одной характеристики материала магнита используют остаточную намагниченность $М$. По определению, намагниченность равна 
объёмной плотности магнитного момента, поэтому для однородно намагниченного шара
\begin{equation}
    m = MV
\end{equation}
\label{eight}

где $V = \frac{4\pi}{3}R^3$ - объём магнита. Величину $B_r = \mu_0 M $ называют остаточной индукцией материала (в ед. СГС $B_r = 4\pi M$).
Из (2) нетрудно видеть, что индукция Вр на полюсах однородно намагниченного шара направлена по нормали к поверхности и совпадает
поэтому с индукцией внутри шара $B_p = B_0 $. Как следует из (7), величина $B_p$ связана с остаточной индукцией $B_r$, соотношением

\begin{equation}
    B_p = B_0 = \frac{2}{3} B_r
\end{equation}
\label{ostind}

\section{Задание}

\textbf{Определение магнитного момента магнитных шариков}
\textbf{Метод А}. 
\begin{figure}[h!]
	\begin{center}
		\includegraphics[width = 0.2\textwidth]{task1.png}
		\caption{Измерение магнитных моментов шариков}
		\label{fig:facility}
	\end{center}
\end{figure}
Величину магнитного момента $\mathfrak{m}$ двух одинаковых шариков можно рассчитать, зная их массу $m$ и определив максимальное расстояние $r_{max}$, на котором они ещё
удерживают друг друга в поле тяжести (см. рис. 1). При максимальном расстоянии сила тяжести шариков $mg$ равна силе их магнитного притяжения. Когда векторы 
двух магнитных моментов ориентированы вертикально, из (6) имеем
\begin{equation}
    \mathfrak{m} = \sqrt{\frac{mgr_{max}^4}{6}}
\end{equation}

По величине $\mathfrak{m}$ с помощью (2) можно рассчитать величину индукции $B$ вблизи любой точки на поверхности шара радиуса $R$. Максимальная величина 
индукции наблюдается на полюсах (см. (9)).
\begin{enumerate} 
    \item Взвесим 10 шариков на весах, рассчитаем массу одного шарика: $m = (0.8271 \pm 0.0001)\text{грамм}$
    \item С помощью микрометра измеряем диаметр шарика: $(6.3 \pm 0.1) \text{мм}$.
    \item С помощью магнитометра измеряем индукцию поля на полюсах шарика $B = (260 \pm 1) \text{мТл}$, $\mathfrak{m} = (410 \pm 40) \text{ед. СГС} $
    \item Проложим между двумя шариками деревянный брусок. Измеряем максимальную толщину бруска, при которой шарики все еще притягивают друг друга:
$r_{max} + d = 2.3 \pm 0.1 \text{см}$  
    \item Расситаем величину магнитного момента 
    \begin{equation}
        \varepsilon_{\mathfrak{m}} = \sqrt{\frac{1}{4} \cdot {\varepsilon_{m}^2} + 4 \cdot {\varepsilon_{r}^2}}
    \end{equation}
    \begin{equation}
        \mathfrak{m} = \sqrt\frac{mgr^4}{6} = (61 \pm 4) (\text{ед. СГС})
    \end{equation} 

    \item Рассчитаем индукцию $B$ вблизи полюсов: $B = \frac{2\mathfrak{m}}{R^3} = (480 \pm 80) (\text{мТл})$
\end{enumerate}

\textbf{Метод Б}.
\begin{figure}[h!]
	\begin{center}
		\includegraphics[width = 0.2\textwidth]{task2.png}
		\caption{Альтернативный метод измерения магнитных моментов шариков}
		\label{fig:facility}
	\end{center}
\end{figure}
    Величину магнитного момента шариков можно определить также по силе их сцепления. Она определяется как сила, необходимая для разрыва двух сцепившихся магнитных шариков. 
Сила сцепления максимальна, если шары соединяются своими противоположными полюсами (магнитные моменты сонаправлены).
    Максимальную силу сцепления можно определить по весу магнитной цепочки, которую способен удержать самый верхний магнитный шарик. Если цепь состоит из 
одинаковых магнитных шариков (см. рис. 2а), то при определённой длине она отрывается от верхнего шарика. При этом, учитывая, что сила притяжения 
убывает как $F \propto 1/r^4$, где $r$ - расстояния между центрами шаров, для расчёта прочности цепочки достаточно учитывать силу взаимодействия 
верхнего шара с 3-4 ближайшими соседями.

    Сила сцепления (6) двух одинаковых шаров радиусами $R$ с магнитными моментами $m$ равна
\begin{equation}
    F_0 = \frac{6m^2}{(2R)^4} = \frac{3m^2}{8R^4} \text{(ед. СГС)}
\end{equation} 
\label{R}

\begin{enumerate}
    \item Составим цепочку из 25 шариков, и с помощью неодимовых магнитов присоединим к цепочке гирю. Подберем минимальный вес системы, при котором она отрывается от верхнего шарика
    \item Вес системы определим с помощью весов $m_{\text{сист}} = (350,652 \pm 0.001) \text{г}$
    \item Рассчитаем силу сцепления, определим по ней магнитный момент: 
    \begin{equation}
        F = 1,08 \cdot F_0 = 3.73 H 
    \end{equation}
    \begin{equation}
        \mathfrak{m} = \sqrt{\left(\frac{8F_0R^4}{3}\right)} = (98 \pm 6) \text{(ед. СГС)}
    \end{equation}
    \item Рассчитаем величину намагниченности материала шариков $M$ и остаточную индукцию магнитного поля $B_r$ Табличное значение $B_r$ для магнита неодим-железо-бор равняется 1,22 Тл. 
    \begin{equation}
        M = \mathfrak{m} / V 
    \end{equation}
    \begin{equation}
        M_a = (471 \pm 29) \text{ед. СГС}
    \end{equation}
    \begin{equation}
        M_b = (750  \pm 45) \text{ед. СГС}
    \end{equation}
    \begin{equation}
        B_r = 4\pi M
    \end{equation}
    \begin{equation}
        B_{r1}  = (0,59 \pm 0.04) \text{Тл}
    \end{equation}
    \begin{equation}
        B_{r2} = (0.94 \pm 0.06) \text{Тл}
    \end{equation}
    Значения в первом эксперименте далеки от табличных, во втором ближе, но все еще не совпадают в пределах погрешности. 
    \item Рассчитаем значение $B_p$
    \begin{equation}
        B_p = \frac{2}{3}B_r
    \end{equation}
    \begin{equation}
        B_p1 = (0,39 \pm 0.02) \text{Тл}
    \end{equation}
    \begin{equation}
        B_p2 = (0,63 \pm 0.04) \text{Тл}
    \end{equation}
\end{enumerate}
\textbf{Измерение горинтальной составляющей индукции магнитного поля Земли}
\begin{figure}[h!]
	\begin{center}
		\includegraphics[width = 0.2\textwidth]{task3.png}
		\caption{Альтернативный метод измерения магнитных моментов шариков}
		\label{fig:facility}
	\end{center}
\end{figure}
    Магнитное поле Земли в настоящей работе измеряется по периоду крутильных колебаний магнитной стрелки вокруг горизонтальной оси.
«Магнитная стрелка» образована сцепленными друг с другом и намагниченными шариками. С помощью А-образного подвеса стрелка подвешена в 
горизонтальном положении (см. рис. 3). Для крепления нити в работе используется штатив, изготовленный из немагнитного материала.
Магнитные моменты всех шариков направлены в одну сторону вдоль оси «стрелки». Под действием механического момента сил (3), действующего на 
стрелку со стороны поля Земли, стрелка стремится повернуться по горизонтальной составляющей магнитно- го поля Земли В в направлении Юг-Север. 
    При отклонении стрелки на угол $\theta$ от равновесного положения в горизонтальной плоскости возникают крутильные колебания вокруг 
вертикальной оси, проходящей через середину стрелки. Если пренебречь упругостью нити, то уравнение крутильных колебаний такого маятника 
определяется возвращающим моментом сил.
\[
\mathcal{M} = - m_n B_{\parallel} \sin{\theta}
\]
и моментом инерции \(J_n\) «стрелки» относительно оси вращения. При малых амплитудах (\(\sin \theta \approx \theta\)) уравнение колебаний стрелки имеет вид
\[
J_n \ddot{\theta} + m_n B_{\parallel} \theta = 0.
\]
Отсюда находим период малых колебаний:
\[
T = 2\pi \sqrt{\frac{J_n}{m_n B_{\parallel}}} \tag{13}
\]
Здесь \(m_n = n m\) — полный магнитный момент магнитной «стрелки», составленной из \(n\) шариков. Момент инерции \(J_n\) стрелки из \(n\) шариков с хорошей точностью равен моменту инерции тонкого однородного стержня массой \(m_n = n m\) и длиной \(\ell_n = 2R\):
\[
J_n \approx \frac{1}{12} m_n \ell_n^2 = \frac{1}{3} m_n R^2.
\]
Отсюда находим, что период колебаний маятника пропорционален числу паров \(n\), составляющих «стрелку»:
\[
T_n = 2\pi \sqrt{\frac{m R^2}{3 m B_{\parallel}}} \cdot \sqrt{n}. \tag{15}
\]


\begin{enumerate}
    \item Соберем крутильный маятник из 12 магнитных шариков и подвесим его на немагнитном штативе. Используя Л-образны подвес, установим магнитную стрелку в горизонтальное положение 
    \item Возбудим крутильные колебания маятника вокруг вертикальной оси и определим их период. Оценим влияние упругости нити на период колебаний, возбудив крутильные колебания стрелки, свернутой в кольцо.
    Убедимся, что упругость нити при расчете периода колебаний можно не учитывать. $T = 4.8 \text{c}$ такой большой период колебаний говорит о том, что упругость нити можно не учиытвать.
    \item Исследуем зависимость периода $T$ крутильных колебаний стрелки от количества магнитных шариков $n$.
    \\
    \begin{tabular}{|c|c|c|c|c|c|c|c|c|c|} \hline
        n & 12 & 11 & 10 & 9 & 8 & 7 & 6 & 5 & 4\\\hline
        t, c & 37.8 & 35.3 & 33.7 & 30 & 27.3 & 24.3 & 20.6 & 17.5 & 14.1\\ \hline
        T, c & 3.78 & 3.53 & 3.37 & 3 & 2.73 & 2.43 & 2.06 & 1.75 & 1.41\\ \hline
    \end{tabular}
    \\
    \item Построим график экспериментальной $T(n)$. По значению углового коэффициента рассчитаем величину горизонтальной составляющей магнитного поля Земли. 
    \\
    \begin{figure}[h!]
        \begin{center}
            \includegraphics[width = 0.6\textwidth]{T(n).png}
            \caption{График зависимости $T(n)$}
            \label{fig:facility}
        \end{center}
    \end{figure} 
    \item График получился линейный, коэффициент наклона $k = (0.30017 \pm 0.0087) c$
    \begin{equation}
        B_{\parallel} = \frac{4\pi^2 mR^2}{3\mathfrak{m}k^2} = (0.12 \pm 0.02) \text{Гс}
    \end{equation}
    \begin{equation}
        \varepsilon_{B_{\parallel}} = \sqrt{{\varepsilon_{m}^2} + 4\cdot {\varepsilon_{R}}^2 + {\varepsilon_{\mathfrak{m}}}^2 + 4\cdot {\varepsilon_{k}}^2 }
    \end{equation}
\end{enumerate}

\textbf{Измерение вертикальной составляющей индукции магнитного поля Земли. Магнитное наклонение.}
Для измерения вертикальной $B_{\perp}$ составляющей вектора индукции поля Земли используется та же установка, что и для измерения горизонтальной 
составляющей с тем лишь отличием, что подвешенная магнитная «стрелка» закрепляется на нити в одной точке. В этом случае 
стрелка, составленная из чётного числа одинаковых шариков и подвешенная за середину, расположится не горизонтально, а под некоторым
углом к горизонту (см. рис. 4а). Это связано с тем, что вектор $B$ индукции магнитного поля Земли не горизонтален, а образует с горизонтом 
некоторый угол $\beta$, зависящий от географической широты $\phi$места, где проводится опыт. Величина угла $\beta$ называется магнитным наклонением.
\begin{figure}[h!]
	\begin{center}
		\includegraphics[width = 0.6\textwidth]{task4.png}
		\caption{Измерение вертикальной составляющей поля и магнитного наклонения}
		\label{fig:facility}
	\end{center}
\end{figure}

Измерить магнитное наклонение непосредственно по положению подвешенной «стрелки» затруднительно из-за механического момента нити в точке подвеса, 
неизбежно возникающем при наклоне «стрелки». Избавиться от этого можно, если выровнять её горизонтально с помощью небольшого дополнительного 
грузика (см. рис. 46). В этом случае момент силы тяжести груза относительно точки подвеса будет равен моменту сил, действующих на «стрелку» 
со стороны вертикальной составляющей магнитного поля Земли. Если масса уравновешивающего груза
равна $m_{\text{гр}}$, плечо силы тяжести $r_{\text{гр}}$ а полный магнитный момент стрелки $m_n = nm$,
то в равновесии
\begin{equation}
    M_n = m_{\text{гр}} g r_{\text{гр}} = nmB_{\perp}
\end{equation}
Видно, что момент $M_n$ силы тяжести уравновешивающего груза пропорционален числу и шариков, образующих магнитную «стрелку»: $M \propto n$.
\begin{enumerate}
    \item Изготовим магнитную стрелку из 10 шариков и подвесим ее за середину на штативе
    \item Определим механический момент сил, действующией со стороны магнитного поля Земли на горизонтально расположенную магнитную стрелку. 
    Для этого, с помощью одного или нескольких кусочков проволоки, уравновесим стрелку в горизонтальном положении.
    С помощью весов определим массу уравновешивающего груза $m_{гр}$
    \item Рассчитаем механический момент сил $\mathcal{M}$, действующих на горизонтальную стрелку со стороны Земли.($\mathcal{M} = n\cdot m$)
    \\
    \begin{tabular}{|c|c|c|c|c|c|c|c|c|c|} \hline
        $n$    & 10   & 12   & 8    & 6    & 4    \\ \hline
        $d$, шар & 3    & 4    & 2    & 2    & 1    \\ \hline
        $m$, г   & 0.284 & 0.284 & 0.255 & 0.189 & 0.237 \\ \hline
        $\mathcal{M}$    & 0.536 & 0.716 & 0.32  & 0.238 & 0.149 \\ \hline
    \end{tabular}

    \begin{figure}[h!]
        \begin{center}
            \includegraphics[width = 0.6\textwidth]{M(n).png}
            \caption{Измерение вертикальной составляющей поля и магнитного наклонения}
            \label{fig:facility}
        \end{center}
    \end{figure}
    \begin{equation}
        B_{\perp} = \frac{k}{\mathfrak{m}} = (0.58 \pm 0.05) \text{Гс}
    \end{equation}
    \begin{equation}
        \varepsilon_{B_{\perp}} = \sqrt{\varepsilon_{k} + \varepsilon_{\mathfrak{m}}}
    \end{equation}
    \begin{equation}
        \beta = arctan \left(\frac{B_{\perp}}{B_{\parallel}}\right) = (80 \pm 10)^{\circ}
    \end{equation}
    \begin{equation} 
        \varepsilon_{\beta} = \sqrt{{\varepsilon_{B_{\perp}}}^2 + {\varepsilon_{B_{\parallel}}}^2}
    \end{equation}

    Рассчитаем теоретическое значение $\beta$ на широте Москвы $\varphi = 56^{\circ}$ в предположении, что Земля - однородно намагниченный шар:
	
	\[\beta = \arcctg{\frac{\frac{2P_m \sin{\varphi}}{R^3}}{\frac{P_m \cos{\varphi}}{R^3}}} \approx 71^{\circ}\]
	
    Полученное значение совпадает с теоретическим в пределах погрешности
\end{enumerate}
\end{document}